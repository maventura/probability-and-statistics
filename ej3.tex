



\begin{section}{Ejercicio 3}
Utilizando $b = 1$, genere una muestra de tamaño $n = 15$. Calcule cada uno de los estimadores con la muestra obtenida y reporte el valor y error de cada estimador.

\begin{subsection}{Resolución}~\\
Comenzaremos implementando, como función, los dos estimadores restantes.


\begin{verbatim}
Bmom<-function(X)
{
mean = 0
  for(i in 1:length(X))
    { 
      mean = mean + X[i];
    }
    mean = mean/length(X)
  return(2*mean)
}

Bmv<-function(X)
{
  return(max(X))
}
\end{verbatim}


Bmom es la forma algorítmica en R de la expresión del estimador calculado en el primer punto, mientras que la función nativa de R, max, nos provee de una forma fácil para calcular el estimador Bmv. Se la deja aun así dentro de una función por consistencia.\\
\\
A continuación se muestra un ejemplo de una corrida en la cual se utiliza la función runif, nativa de R, la cual genera números flotantes aleatorios siguiendo una distribución uniforme, para generar la muestra. Luego de eso realiza  la estimación de b mediante Bmom, Bmv y Bmed.


\begin{verbatim}

> X = runif(15, 0.0, 1.0)
> X
 [1] 0.85100376 0.35369725 0.37211795 0.02894966 0.80028852
     0.73197841 0.24412642 0.84706643 0.74270632 0.85146423
     0.65617966 0.11617666 0.11925780 0.25546167 0.53364961
> Bmed(X)
[1] 0.7442359
> Bmv(X)
[1] 0.8514642
> Bmom(X)
[1] 1.00055


\end{verbatim}

Utilizaremos ahora R para calcular también los errores

\begin{verbatim}
> abs(1-Bmom(X))
[1] 0.0005499117
> abs(1-Bmv(X))
[1] 0.1485358
> abs(1-Bmed(X))
[1] 0.2557641
> 
\end{verbatim}

Como puede verse, los valores obtenidos por los estimadores Bmv, Bmom y Bmed son 0.8514642, 1.00055 y 0.7442359 respectivamente, y sus errores son 0.1485358, 0.0005499117 y 0.2557641.


\end{subsection}
\end{section}