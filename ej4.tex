



\begin{section}{Ejercicio 4}
Hacer una simulación para obtener el sesgo, varianza y error cuadrático medio (ECM) de
cada uno de los estimadores.

\begin{subsection}{Sesgo}~\\

\begin{verbatim}
b = 1;
n = 15;
nrep = 1000;
Bmoms = double()
Bmeds = double()
Bmvs = double()
X = double()

for (i in 1:nrep){
 X = runif(n, 0.0, b);
 val1 = 
 Bmoms[i] <- Bmom(X);
 Bmeds[i] <- Bmed(X);
 Bmvs[i] <- Bmv(X);
}
momMean = mean(Bmoms);
medMean = mean(Bmeds);
mvMean = mean(Bmvs);

momErr = abs(b-momMean);
medErr = abs(b-medMean);
mvErr = abs(b-mvMean);

\end{verbatim}
Los resultados conseguidos son los siguientes:

\begin{verbatim}

> print(momErr)
[1] 0.0002888454
> print(medErr)
[1] 0.1260778
> print(mvErr)
[1] 0.06227115
> 
\end{verbatim}
como puede verse, los resultados fueron:\\
~\\
 (momErr, medErr, mvErr) = (0.0002888454, 0.1260778, 0.06227115).\\
~\\
\end{subsection}
\begin{subsection}{Varianza}~\\

Las varianzas muestrales de los estimadores son las siguientes:


\begin{verbatim}

> momVar = var(Bmoms)
> medVar = var(Bmed)
Error: is.atomic(x) is not TRUE
> medVar = var(Bmeds)
> mvVar = var(Bmvs)
> print(momVar)
[1] 0.02233229
> print(medVar)
[1] 0.05594595
> print(mvVar)
[1] 0.003525818
> 

\end{verbatim}

Por lo tanto, al usar la varianza muestral de los estimadores  Bmom, Bmed y Bmv, la aproximación obtenida para varianza da: \\
~\\
\center{(momVar, medVar, mvVar) = (0.02233229, 0.05594595, 0.003525818)}~\\
~\\

\end{subsection}
\begin{subsection}{Error cuadrático medio}~\\

La formula que relaciona el error cuadrático medio con el sesgo y la varianza es \\
~\\
~\\
$ECM_B(\hat{B}) = V_B(\hat{B}) + (E_B(\hat{B}) - B)^2$\\
~\\
~\\
Utilizaremos esta formula para calcular, para cada estimador, el ECM como se muestra en el siguiente código:

\begin{verbatim}
> ECMBmom = momVar + (momErr^2);
> ECMBmed = medVar + (medErr^2);
> ECMBmv  = mvVar + (mvErr^2);
> print(ECMBmom)
[1] 0.02233238
> print(ECMBmed)
[1] 0.07184158
> print(ECMBmv)
[1] 0.007403514
> 
\end{verbatim}

Como puede verse, los errores cuadráticos medios, calculados así, resultan ser:\\
~\\
(ECMBmom, ECMBmed, ECMBmv) = (0.02233238, 0.07184158, 0.007403514).
~\\
\end{subsection}
\end{section}