
\begin{section}{Ejercicio 1}
~\\


Sean $X_1, ... ,X_n$ una muestra aleatoria con distribución $U(0, b)$ con b un parámetro desconocido. Calcular analíticamente los estimadores de momentos $\hat{b}_{mom}$ y de máxima verosimilitud $\hat{b}_{mv}$. Implementarlos en R como funciones.


\begin{subsection}{Estimador de momentos}
~\\

Comenzaremos calculando la esperanza de una variable uniforme de parámetros 0 y b.\\
$E(U) = \frac{b+0}{2} = \frac{b}{2}.$\\
~\\

Ahora para la esperanza muestral, obtenemos\\
~\\

$\frac{X_1+ ... +X_n}{n}$. \\
~\\
Igualamos.\\
~\\

  $\frac{X_1+ ... +X_n}{n} = \frac{b}{2}$\\
 $2\frac{X_1+ ... +X_n}{n} = \hat{b}_{mom}$\\
\end{subsection}

\begin{subsection}{Estimador de máxima verosimilitud}
~\\

Tomo $f(X_1, ... ,X_n) = \prod_{i=1}^{n}{f(x_i)} = $ 
$\prod_{i=1}^{n}{\frac{1}{b} I_{[0,b]}(x_i)}$ \\
~\\

Con lo cual se obtiene la siguiente función de verosimilitud:\\
~\\

$L(b) = \frac{1}{b^n} \prod_{i=1}^{n}{ I_{[0,b]}(x_i)}$\\
~\\

Notemos que la función también se puede definir como $\frac{1}{b^n}$ si $0 < X_i < b$ $ \forall i$, o cero fuera de ese rango. O sea, si algún $X_i$ es mayor a b, la función se vuelve nula. \\
Pero esta condición, $0 \leq X_i \leq b$ $ \forall i$ puede pensarse como $ b > max(X_i)$.\\
~\\

Dado que donde no es nula, la función es monótona decreciente, puesto que esta compuesta por un cociente con un denominador positivo y monótono creciente, el máximo debe encontrarse exactamente en el punto en el cual deja de ser nula. Esto es, $max(X_i)$, y por lo tanto, este es el estimador de máxima verosimilitud.\\
~\\

$\hat{b}_{mv} = max(x_i)$
~\\

\end{subsection}


\end{section}


