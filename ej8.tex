



\begin{section}{Ejercicio 8}
Calcular los estimadores en la siguiente muestra. ¿Observa algo extraño? ¿a que cree que se
debe? \\
~\\
0,917 0,247 0,384 0,530 0,798 0,912 0,096 0,684 0,394 20,1 0,769 0,137 0,352 0,332 0,670

\begin{subsection}{Resolución}~\\


Se utiliza el siguiente código para generar los vectores.
\begin{verbatim}

> vals = double();
> vals <- c(0.917, 0.247, 0.384, 0.530, 0.798, 0.912, 0.096, 0.684, 0.394, 20.1, 0.769, 0.137, 0.352, 0.332, 0.670)
> Rmed = Bmed(vals)
> Rmom = Bmom(vals)
> Rmv = Bmv(vals)
> print(Rmom)
[1] 3.642933
> print(Rmv)
[1] 20.1
> print(Rmed)
[1] 0.192
\end{verbatim}
\end{subsection}
\begin{subsection}{Análisis}~\\

Los resultados de los estimadores son (Bmom, Bmed, Bmv) = (3.642933, 0.192, 20.1). Es esperable que esto suceda ya que el estimador $B_{mv}$, estima utilizando el máximo. Si este fuera un outlier resultante de un error de medición, nos estaría dando un muy mal resultado. Lo que esto nos dice es que este método es mas sensible a errores de medición.


\end{subsection}
\end{section}